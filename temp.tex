\documentclass{article}
\usepackage{tikz}
\usetikzlibrary{matrix,chains,positioning,decorations.pathreplacing,arrows}

\begin{document}

\begin{tikzpicture}[
plain/.style={
  draw=none,
  fill=none,
  },
net/.style={
  matrix of nodes,
  nodes={
    draw,
    circle,
    inner sep=10pt
    },
  nodes in empty cells,
  column sep=2cm,
  row sep=-9pt
  },
>=latex
]
\matrix[net] (mat)
{
|[plain]| \parbox{1.3cm}{\centering Input\\layer} & |[plain]| \parbox{1.3cm}{\centering Hidden\\layer} & |[plain]| \parbox{1.3cm}{\centering Output\\layer} \\
& |[plain]| \\
|[plain]| & \\
& |[plain]|  \\
  |[plain]| & |[plain]| \\
& & \\
  |[plain]| & |[plain]| \\
& |[plain]| \\
  |[plain]| & \\
& |[plain]| \\    };
\foreach \ai [count=\mi ]in {2,4,...,10}
  \draw[<-] (mat-\ai-1) -- node[above] {Input \mi} +(-2cm,0);
\foreach \ai in {2,4,...,10}
{\foreach \aii in {3,6,9}
  \draw[->] (mat-\ai-1) -- (mat-\aii-2);
}
\foreach \ai in {3,6,9}
{\foreach \aii in {3,6,9}
  \draw[->] (mat-\ai-2) -- (mat-\aii-3);
};
\end{tikzpicture}


\end{document}